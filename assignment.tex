\documentclass{article}

\usepackage{amssymb}
\usepackage{graphicx}
\usepackage{amsmath}
\usepackage{amsthm}
\usepackage{hyperref}
\usepackage{fancyhdr}
\newtheorem{conjecture}{Conjecture}[section]
\newtheorem{theorem}{Theorem}[section]
\pagestyle{fancy}
\fancyhf{}  % Clear header and footer
\usepackage[a4paper, margin=0.5in]{geometry}
\renewcommand{\headrulewidth}{0pt} 
\fancyhead[R]{\thepage} 
\hypersetup{
	colorlinks=true,
	linkcolor=blue,
	urlcolor=blue,
	citecolor=blue,
%	pdfborderstyle={/S/U/W 1},
}



\title{\textbf{THE 18.821 MATHEMATICS PROJECT LAB REPORT
		\newline[REPLACE THIS WITH YOUR OWN SHORT DESCRIPTIVE TITLE!]}}
\author{X. BURPS, P. GURPS}
\date{}

\makeatletter
\renewcommand\section{\@startsection {section}{1}{\z@}%
	{-3.5ex \@plus -1ex \@minus -.2ex}%
	{2.3ex \@plus.2ex}%
	{\normalfont\Large}}
\makeatother
\begin{document}
	\maketitle
	ABSTRACT. This is a \LaTeX{} template for 18.821, which you can
	 use for your own reports.
	 \centering         \section{INTRODUCTION}
	This brief document shows some examples of the use of \LaTeX and
	indicates some special features of the Math Lab report style. The
	\texttt{\href{http://stellar.mit.edu/S/course/18/sp13/18.821/}{\underline{course website}}}contains links to several \LaTeX manuals.
	End the introduction by describing the contents of the paper sec­
	tion by section, and which team member(s) wrote each of them. For
	instance, Section \textcolor{blue}{6} discusses referencing, and is written by P. Gurps.
	\section{\LaTeX EXAMPLES}
	Here are some ways of producing mathematical symbols. Some are pre-defined either in \LaTeX or in the
	package which this document loads.For instance sums and integrals, $ \sum_{i=1}^{n} 1=n ,\int^a_b xdx =\frac{n^2}{2}.$ We’ve defined a few other symbols at the start of the document, forinstance $\mathbb{N,Q,Z,R}$ You can make marginal notes for yourself or your
	co-authors like this:	Unfinished here?
	If you want to typeset equations, there are many choices, with or
	without numbering:
\[	\int^0_1 xdx= \frac{1}{2}, \]
\raggedright 
or  
\[	\sum_{i=1}^{\infty}i=\frac{-1}{12} \]
\raggedright
or
\[  1-1+1-...=\frac{1}{2}.\]
\date{Date:February 10, 2013.}
\newpage
\centering {X. BURPS, P. GURPS}
\begin{figure}[h]
\centering
\includegraphics[width=0.7\textwidth]{pic.png}
\caption{My first pdf figure.}
\label{fig:pic}
\end{figure}
\begin{align}
	\lim_{n \to \infty} \sum_{k=1}^{n} \frac{1}{k^2}=\frac{\pi}{6}.
\end{align} 
This can then be referred to as(\textcolor{blue}{1}), which is much easier than keeping
track of numbers by hand. To group several equations, aligning on the
= sign, do it like this:
\begin{align*}
	x_1 + 2x_2 + 3x_3 &= 7 \\
	y &= mx + c \\
	&= 4x - 9.
\end{align*}
You can easily embed hyperlinks into the output .pdf document:
\texttt{\href{http://stellar.mit.edu/S/course/18/sp13/18.821/}{\underline{click here}}} for example.
\section{IMAGES}
Figure \textcolor{blue}{1} is an example of a .pdf image put into a floating environ­
ment, which means LaTeX will draw it wherever there’s enough space
left in your manuscript. Look at the .tex original to see how to insert
a figure like this.
\section{THEOREMS AND SUCH}
An example of a “conjecture environment” is given below, in Con­jecture \textcolor{blue}{4.1}. Theorems, lemmas, propositions, definitions, and such all
use the same command with the appropriate name changed. In fact,
\newpage \section*{THE 18.821 REPORT}if you look at the top of this .tex file, you can see where we’ve defined these environments.\\
\begin{conjecture}[Vaught's Conjecture]Let $T$ be a countable complete theory. If $T$ has fewer than $2^{\aleph_0}$ many countable models (up to isomorphism), then it has countably many countable models.
\end{conjecture}

\begin{theorem}
	When it rains it pours.
\end{theorem}
\begin{proof}
	Well, yes.
\end{proof}
\section{FILETYPES USED BY LATEX}
You will write your text as  a \texttt{.tex} file using any text editor (though WYSIWYG editors are troublesome). Traditionally one then runs \LaTeX{} and obtains a \texttt{.dvi} file, which can be viewed on the screen using a dvi viewer. To include images, and then prepare the file for printing or submission, one typically translates the \texttt{.dvi} into either \texttt{.ps} (Postscript) or \texttt{.pdf} (Adobe PDF).
\par{}\hspace{0.5 cm}Your report will be submitted as a \texttt{.pdf} document. The \texttt{pdflatex} command produces a \texttt{.pdf} file directly from a \texttt{.tex} file. This command works well with included \texttt{.pdf} files, but does not handle \texttt{.eps} files. An \texttt{.eps} file can be converted to a \texttt{.pdf} file by viewing it and saving as a \texttt{.pdf} file, or by \texttt{ps2pdf filename.eps}, which produces \texttt{filename.pdf}.Under MikTeX with WinEdt, all necessary commands will appear under “Accessories” in the WinEdt me.
\par{}\hspace{0.5 cm} Finally, Matlab can be made to produce \texttt{.eps} files by typing\\
\hspace{3 cm} \texttt{print -deps filename}\\ at the prompt.
\section{QUOTING SOURCES}
In your work, keep notes of the literature you’ve used, including websites. Cite the references you use; failure to do so constitutes plagiarism. Every bibliography item should be referenced somewhere in the paper. Quote as precisely as possible:[\textcolor{blue}{1},pages 76--78] rather than~\cite{itref}.~\cite{itref2} was a useful background reference, too.

\begingroup \renewcommand{\refname}{REFERENCES}
\begin{thebibliography}{9}
	\bibitem{itref}
	Gurps, P., \emph{Care and feeding of maths professors}. Cambridge Univ. Press, 2008.
	\bibitem{itref2}
	Burps, X. \emph{Terrors and errors of project lab}. \emph{Journal of Wildlife and Conservation} 21 (2008), 112--134.
\end{thebibliography}
\endgroup
\appendix

\section*{APPENDIX}
Appendices are useful for putting in code or data.
\newpage
\raggedright
MIT OpenCourseWare 
\newline
\href{http://ocw.mit.edu}{\underline {http://ocw.mit.edu}}
\vspace{2 cm}
\newline
18.821 Project Laboratory in Mathematics \\
Spring 2013 
\vspace{2 cm}
\newline
For information about citing these materials or our Terms of Use, visit:
\href{http://ocw.mit.edu/terms}{\underline {http://ocw.mit.edu/terms.}}
\end{document}